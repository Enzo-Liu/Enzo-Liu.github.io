% Created 2016-02-14 Sun 10:50
\documentclass[presentation, bigger]{beamer}
\usepackage[no-math]{fontspec}
\usepackage[BoldFont,SlantFont,AutoFakeBold=true,AutoFakeSlant=true]{xeCJK}
\setCJKmainfont[BoldFont=FandolSong-Bold.otf,ItalicFont=FandolKai-Regular.otf]{FandolSong-Regular.otf}
\setCJKsansfont[BoldFont=FandolHei-Bold.otf]{FandolHei-Regular.otf}
\setCJKmonofont{FandolFang-Regular.otf}
\usefonttheme[stillsansseriflarge,stillsansserifsmall]{serif}
\usepackage{graphicx}
\usepackage{xcolor}
\usepackage{listings}
\defaultfontfeatures{Mapping=tex-text}
\usepackage{geometry}
\usepackage{verbatim}
\usepackage{fixltx2e}
\usepackage{longtable}
\usepackage{float}
\usepackage{wrapfig}
\usepackage{rotating}
\usepackage[normalem]{ulem}
\usepackage{amsmath}
\usepackage{marvosym}
\usepackage{wasysym}
\usepackage{amssymb}
\usepackage{hyperref}
\tolerance=1000
\usetheme{metropolis}
\author{刘恩泽}
\date{2016-02-02}
\title{拼团活动技术方案}
\hypersetup{
 pdfauthor={刘恩泽},
 pdftitle={拼团活动技术方案},
 pdfkeywords={},
 pdfsubject={},
 pdfcreator={Emacs 24.5.1 (Org mode 8.3.3)},
 pdflang={English}}
\begin{document}

\maketitle
\begin{frame}{目录}
\tableofcontents
\end{frame}



\section{活动分析}
\label{sec:orgheadline2}

\begin{frame}[label={sec:orgheadline1}]{活动分析}
\begin{enumerate}
\item 针对活动演出,提供两个商品,一个原价商品,一个团购商品
\item 对于团购商品的购买,订单支付之后,需要延迟发货相关的流程。
\item 团购商品的订单,支付后,可以获取分享的团购标识.
\item 在时效范围内且参团人数未满,都可以参加该团购的购买
\item 一次开团成功后,所有相关订单开始备货发货流程
\item 一次开团失败后,所有相关订单走退款流程
\end{enumerate}
\end{frame}


\section{功能列表}
\label{sec:orgheadline5}

\begin{frame}[label={sec:orgheadline3}]{接口交互}
\begin{enumerate}
\item 活动演出列表
\item 活动演出详情
\item 参团详情
\item 获取团购标识
\end{enumerate}
\end{frame}


\begin{frame}[label={sec:orgheadline4}]{后台逻辑}
\begin{enumerate}
\item 标记特卖商品 (普通售卖不可见)
\item 配置团购商品的汇总关系 (列表页数据)
\item 购买时根据团购商品区分不同的发货
\item 根据团购订单生成开团标识
\item 定期根据团购标识处理发货或者退款
\end{enumerate}
\end{frame}


\section{后台接口列表}
\label{sec:orgheadline6}
\begin{enumerate}
\item 根据活动ID拿商品列表
\item 根据商品拿详情
\item 根据用户和商品获取拼团信息
\item 参加/发起一个拼团
\end{enumerate}

\section{生命周期}
\label{sec:orgheadline13}

\begin{frame}[label={sec:orgheadline7}]{创建}
\begin{itemize}
\item 创建团
\item 加入团
\end{itemize}
\end{frame}

\begin{frame}[label={sec:orgheadline8}]{支付}
\begin{itemize}
\item 加入团成功
\end{itemize}
\end{frame}

\begin{frame}[label={sec:orgheadline9}]{订单关闭}
\begin{itemize}
\item 加入团失败
\end{itemize}
\end{frame}

\begin{frame}[label={sec:orgheadline10}]{订单退款}
\begin{itemize}
\item 加入团失败
\end{itemize}
\end{frame}

\begin{frame}[label={sec:orgheadline11}]{团成功}
\end{frame}

\begin{frame}[label={sec:orgheadline12}]{团失败}
\end{frame}


\section{场景}
\label{sec:orgheadline15}
\begin{frame}[label={sec:orgheadline14}]{页面入口的场景}
\begin{center}
\begin{tabular}{lllll}
人 & 团 & 团状态 & 操作 & 提示\\
\hline
参加过团A & B &  & 不能做任何事 & 已参加过一个团\\
参加过团A & A & 未结束 & 只能分享 & \\
参加过团A & A & 成功 & 不能做任何事 & \\
参加过团A & A & 失败 & 创建新的团 & \\
未参加过 & / & / & 创建新的团 & \\
未参加过 & A & 进行中 & 参加该团 & \\
未参加过 & A & 结束(成功、失败) & 创建新的团 & \\
\end{tabular}
\end{center}
\end{frame}
\end{document}
