% Created 2017-02-07 Tue 11:34
% Intended LaTeX compiler: pdflatex
\documentclass[presentation,bigger]{beamer}
\usepackage[no-math]{fontspec}
\usepackage[BoldFont,SlantFont,AutoFakeBold=true,AutoFakeSlant=true]{xeCJK}
\setCJKmainfont[BoldFont=FandolSong-Bold.otf,ItalicFont=FandolKai-Regular.otf]{FandolSong-Regular.otf}
\setCJKsansfont[BoldFont=FandolHei-Bold.otf]{FandolHei-Regular.otf}
\setCJKmonofont{FandolFang-Regular.otf}
\usefonttheme[stillsansseriflarge,stillsansserifsmall]{serif}
\usepackage{graphicx}
\usepackage{xcolor}
\usepackage{listings}
\defaultfontfeatures{Mapping=tex-text}
\usepackage{geometry}
\usepackage{verbatim}
\usepackage{fixltx2e}
\usepackage{longtable}
\usepackage{float}
\usepackage{wrapfig}
\usepackage{rotating}
\usepackage[normalem]{ulem}
\usepackage{amsmath}
\usepackage{marvosym}
\usepackage{wasysym}
\usepackage{amssymb}
\usepackage{hyperref}
\setlength{\parindent}{0in}
\tolerance=1000
          \AtBeginSection[]{\begin{frame}<beamer>\frametitle{Topic}\tableofcontents[currentsection]\end{frame}}
\usetheme{metropolis}
\author{Enzo Liu}
\date{\today}
\title{下一轮方向}
\hypersetup{
 pdfauthor={Enzo Liu},
 pdftitle={下一轮方向},
 pdfkeywords={},
 pdfsubject={},
 pdfcreator={Emacs 25.1.1 (Org mode 9.0.3)},
 pdflang={English}}
\begin{document}

\maketitle
\begin{frame}{目录}
\tableofcontents
\end{frame}



\section{主旨}
\label{sec:org3b823a7}

\begin{frame}[label={sec:org1ca7132}]{主旨}
\begin{quote}
总算活下去了,想继续活下去,甚至将来活的好一些。
\end{quote}

\begin{itemize}
\item 于是分析一下现在,整理一下后续的方向。
\end{itemize}
\end{frame}

\section{现状}
\label{sec:org9f8b48c}

\begin{frame}[label={sec:org08ccc04}]{做到了}
\begin{itemize}
\item 以二级市场交易切入行业
\item 初步的运营 (活动, 优惠, 折扣, 专场)
\item 初步的商户服务 (客服,配送)
\item 初步的服务保障
\item 积累了一些用户和商户及相关的交易信息
\item 一定的品牌知名度 \dots{}
\end{itemize}
\end{frame}

\begin{frame}[fragile,label={sec:org345d86f}]{当前价值/优势}
 从用户侧\footnote{不提 \texttt{买的到} , 因为高价买哪里都能买到,没有独特的资源。所以提到了合理的价格和补贴。}和商家侧两个用户群体来看。

\begin{columns}
\begin{column}{0.45\columnwidth}
\begin{exampleblock}{用户侧}
\begin{itemize}
\item 交易的保障
\item 购买的决策依据
\item 相对公允的价格
\item 平台的补贴
\end{itemize}
\end{exampleblock}
\end{column}

\begin{column}{0.45\columnwidth}
\begin{exampleblock}{商户侧}
\begin{itemize}
\item 极低的获客成本
\item 售后相关服务
\item 平台的促销服务
\item 渠道资源(B->B)
\end{itemize}
\end{exampleblock}
\end{column}
\end{columns}

\note{补充说明
}
\end{frame}

\section{下一步}
\label{sec:org30af355}

\begin{frame}[label={sec:org64277bc}]{目标\footnote{大的目标需要老板给点念想了 :)}}
\begin{itemize}
\item 活下去
\item 规模扩大
\item 行业扎根
\item 构建门槛 \dots{}
\end{itemize}
\end{frame}

\begin{frame}[label={sec:org1660f6d}]{纵向扩展}
\begin{itemize}
\item 深入理解用户/商品, 增强品牌认知, 降低交易成本, 扩大交易规模
\item 深入客户服务, 用户侧保障(真假, 价格保护, 退换, 转让), 商家侧保障(滞销, 资金, 库存, 价格)
\item 供应链优化\footnote{感觉是规模化的障碍, 所以单独提出}
\end{itemize}
\end{frame}

\begin{frame}[label={sec:org41890f1}]{横向扩展}
深入从制作到销售到演出到结束的所有环节\footnote{以上只是我以为的环节}

\begin{itemize}
\item 优质 IP
\item 制作
\item 报批
\item 场馆
\item 宣传
\item 发行
\item 票务解决方案
\item 反馈,准备下一轮
\item 数据
\end{itemize}
\end{frame}

\begin{frame}[fragile,label={sec:orge188f05}]{其他方向 (渗透?)}
 增加 \texttt{可以提高用户粘性/访问频次} 的内容, 带动低频的票务交易
\end{frame}

\section{交易侧分析}
\label{sec:orge69639f}

\begin{frame}[fragile,label={sec:org34ae129}]{核心目标}
 \begin{itemize}
\item 交易量, 规模
\item 用户买到想买的
\item 商户赚到钱
\item 希望 \texttt{我们也能赚到钱}
\end{itemize}
\end{frame}

\begin{frame}[fragile,label={sec:org1f570da}]{方向}
 \begin{quote}
方向间也会相互补充, 并不是完全正交
\end{quote}
\begin{itemize}
\item 提高 \texttt{品牌信任} \footnote{作为担保交易的主体必须被认为可靠}
\item 降低 \texttt{交易成本} \footnote{信任平台的成本, 沟通的成本, 票券流通的成本,渠道的风险成本 \dots{}}
\item 提高 \texttt{服务质量} \footnote{保障,价格保护,数据服务, 营销服务, 其他增值服务}
\item 扩大 \texttt{访问规模} \footnote{口碑, 推广, 流量, 新城市, 小型独家 \dots{}}
\end{itemize}
\end{frame}

\begin{frame}[label={sec:org875d55f}]{提高品牌信任}
\begin{itemize}
\item 平台规模
\item 用户口碑
\item 品牌宣传
\item 高公信团体背书
\end{itemize}
\end{frame}

\begin{frame}[fragile,allowframebreaks]{降低交易成本}
 首先分析一下交易环节中, \texttt{用户}, \texttt{商户}, \texttt{平台}  三方各自的环节和成本

\begin{columns}
\begin{column}{0.3\columnwidth}
\begin{exampleblock}{用户}
\begin{itemize}
\item 进入平台
\item 查找商品
\item 选价格,卖家
\item 下单支付
\end{itemize}
\end{exampleblock}
\end{column}

\begin{column}{0.3\columnwidth}
\begin{exampleblock}{商户}
\begin{itemize}
\item 选商品挂售
\item 定库存,价格
\item (预先)采购
\item 备货给平台
\end{itemize}
\end{exampleblock}
\end{column}

\begin{column}{0.3\columnwidth}
\begin{exampleblock}{平台}
\begin{itemize}
\item 演出录入
\item 配货发货
\item 例外处理
\item 客服等售后
\end{itemize}
\end{exampleblock}
\end{column}
\end{columns}

\framebreak


\begin{block}{用户成本}
\begin{itemize}
\item 信任平台担保
\item 挑选想要商品的过程
\item 比较价格和商家的过程
\end{itemize}
\end{block}

\framebreak


\begin{exampleblock}{降低用户成本}
\begin{itemize}
\item 增加平台担保的可信度
\item 降低用户选择想要商品的成本 (例如场景化)
\item 增加商品的差异化, 降低比较价格和商家的难度 (能体现出价格的差异点,比如服务,比如位置,比如快递速度,现票/预售票 \dots{})
\end{itemize}
\end{exampleblock}


\framebreak


\begin{block}{商户成本}
\begin{itemize}
\item 选择售卖商品的过程(好卖, 赚钱, 利润 \dots{} )
\item 决定/调整 卖什么价格卖多少份的过程
\item 采购的过程/提前采购好的风险
\item 发货成本
\end{itemize}
\end{block}

\framebreak

\begin{exampleblock}{降低商户成本}
\begin{itemize}
\item 数据服务, 提供商家项目, 价格, 库存的参考, 降低决策的成本/风险
\item 给商家提供系统对接, 直接打票, 与系统同步库存等服务, 降低备货,调整库存的成本
\end{itemize}
\end{exampleblock}


\framebreak



\begin{block}{平台成本}
\begin{itemize}
\item 商品信息录入的成本
\item 为准入的商家负责, 转采等成本
\item 备货, 发货的成本
\item 售后,咨询等服务成本
\end{itemize}
\end{block}

\framebreak


\begin{exampleblock}{降低平台成本}
我觉得核心在于规模化, \texttt{做边际成本能降低的服务}.
\begin{itemize}
\item 系统化商品信息, 通过对接等方式快速录入新内容
\item 电子票
\item 自动化/产品化常见售后类案例, 积累常见客服案例自助咨询.
\end{itemize}
\end{exampleblock}
\end{frame}


\begin{frame}[label={sec:orgc0ed51d}]{提高服务质量}
\begin{block}{保障类}
\begin{block}{用户}
真假, 位置, 到货时间, 价格保护, 退换, 临时有事不去 \dots{}
\end{block}
\begin{block}{商户}
保销, 保量 \dots{}
\end{block}
\end{block}
\begin{block}{增值服务类}
\begin{block}{用户}
会员折扣, 赠票, 年票, 周边, 团体定制 \dots{}
\end{block}

\begin{block}{商户}
促销, 代销, 营销, 解决滞销, 代对接, 联系上游渠道 \dots{}
\end{block}
\end{block}
\end{frame}

\begin{frame}[label={sec:org53a8662}]{扩大访问规模}
\begin{itemize}
\item 资源置换 (尤其得明确平台到底有哪些资源可以用)
\item 推广
\item 口碑
\item 流量
\item 新城市
\item 小型独家
\end{itemize}
\end{frame}


\section{交易侧措施}
\label{sec:org7c38b07}

\begin{frame}[label={sec:orgec7ec44}]{品宣投入}
\begin{itemize}
\item 购买的第一道坎,就是不知道到底靠不靠谱
\item 知道我们的人还是很少
\end{itemize}
\end{frame}

\begin{frame}[label={sec:orgca03f38}]{开源}
\begin{itemize}
\item 开新城市
\item 加强合作,资源置换, 线下场馆合作,小型主办合作, 引流推广
\item 加强口碑推广,宣传运营
\end{itemize}
\end{frame}

\begin{frame}[fragile,label={sec:orge6b8580}]{加强运营}
 我把用户,商户的运营都算在一起。
\begin{block}{目标}
\begin{itemize}
\item 搞明白用户到底需要什么, 解决问题, 提高用户注册, 活跃, 留存, 消费
\item 利用平台资源调动商家挂售积极性
\end{itemize}

\texttt{具体可以做的内容回头专项分析。}
\end{block}
\end{frame}


\begin{frame}[fragile,label={sec:org26afb67}]{商品信息精细化/差异化}
 \begin{quote}
解决不知道有什么不同价格有什么差别的问题
\end{quote}
\begin{quote}
同时, 针对商家, 提供差异化的售卖可能
\end{quote}
比如 位置,区域,距离,视角, 音效 (根据位置做个 \texttt{VR} 感受一下)
以及如何让商户可以方便的录入/系统方便的根据位置生成
\end{frame}

\begin{frame}[label={sec:org6e1f05a}]{电子票解决方案}
做的好,可以解决以下几个问题:

\begin{itemize}
\item 信任成本(验证,鉴真容易)
\item 流通成本 (电子票怎么转让都比快递省吧)
\item 知晓成交价格的分布, 知道市场情况
\end{itemize}

我想象中做好的标准:
\begin{quote}
类似于微信账号中的一张可转让可消费的券
\end{quote}
\begin{itemize}
\item 便于流通,但所有转让需经过平台
\item 安全性
\item 操作的简单
\item 绑定人和位置信息
\end{itemize}
\end{frame}

\begin{frame}[label={sec:orgeb97c21}]{供应链优化}
\begin{itemize}
\item 将在上话,大舞台上面对面的票务置换系统化
\item 长尾打票的模式平台提供代做系统对接的功能
\end{itemize}
\end{frame}

\begin{frame}[label={sec:org647ce00}]{众筹预售}
对于销售不可知, 不敢采购的问题, 也许可以利用众筹包票预售的方式提前试探一下市场.

对于小型项目之类的也可以尝试一下这个模式.
\end{frame}

\begin{frame}[label={sec:orgfa0438d}]{接小型主办试行电子票}
\end{frame}

\begin{frame}[label={sec:org471f12e}]{保障等增值服务}
\begin{itemize}
\item 用户保障: 到手时间保障,跳票赔付,短期价格波动赔付, 退换(当前已有) \dots{}
\begin{itemize}
\item 通过类似保险思路
\end{itemize}
\item 观演服务: 接机,住宿,接送车, 演出周边, 团体包场定制 \dots{}
\item 商家资金服务: 担保贷款(接入 p2p 信贷服务)
\item 促销服务: 商家自定义活动, 平台配合可有资源池补贴
\item 精准营销服务: 给商家提供营销资源
\item 代销服务: 平台联系销售渠道(比如各公司党组织,解决大量滞销票,对方有利益就行)
\item 定期赠票: 增值服务
\item 联合场馆发行年票 \dots{}
\end{itemize}
\end{frame}

\begin{frame}[label={sec:orgfe1285c}]{其他?}
欢迎补充
\end{frame}
\end{document}