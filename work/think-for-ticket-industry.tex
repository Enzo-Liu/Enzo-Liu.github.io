% Created 2016-07-17 Sun 13:20
\documentclass[11pt,a4paper]{article}
\usepackage{fontspec}
\usepackage{xeCJK}
\setCJKmainfont[BoldFont=FandolSong-Bold.otf,ItalicFont=FandolKai-Regular.otf]{FandolSong-Regular.otf}
\setCJKsansfont[BoldFont=FandolHei-Bold.otf]{FandolHei-Regular.otf}
\setCJKmonofont{FandolFang-Regular.otf}
\usepackage{graphicx}
\usepackage{xcolor}
\usepackage{listings}
\defaultfontfeatures{Mapping=tex-text}
\usepackage{geometry}
\usepackage{verbatim}
\usepackage{fixltx2e}
\usepackage{longtable}
\usepackage{float}
\usepackage{wrapfig}
\usepackage{rotating}
\usepackage[normalem]{ulem}
\usepackage{amsmath}
\usepackage{marvosym}
\usepackage{wasysym}
\usepackage{amssymb}
\usepackage{hyperref}
\usepackage{parskip}
\setlength{\parindent}{15pt}
\usepackage{indentfirst}
\geometry{a4paper, textwidth=6.5in, textheight=10in,
            marginparsep=7pt, marginparwidth=.6in}
\tolerance=1000
\pagestyle{empty}
\author{enzo liu}
\date{2016-07-17 Sun}
\title{票务行业的思考}
\hypersetup{
 pdfauthor={enzo liu},
 pdftitle={票务行业的思考},
 pdfkeywords={piaoniu},
 pdfsubject={piaoniu},
 pdfcreator={Emacs 25.0.95.1 (Org mode 8.3.4)},
 pdflang={English}}
\begin{document}

\maketitle
\tableofcontents

工作以来,学会的最重要的一件事情,就是逼问自己,到底在做什么。

这两天,在读一本书叫做  \textbf{rework} , 看见的最有感触的就是相对规范化的提供了几个问题用来拷问自己。记下我觉得最重要的几个问题:

\begin{itemize}
\item why are you doing this?
\item what problems are you solving?
\item is this actually useful?
\item are you adding value?
\end{itemize}

借此也梳理一下这一年来,我对这个行业的理解。

\section*{行业认知 (存在的问题, 回答 why)}
\label{sec:orgheadline2}

从行业的模式来分,主要有两类玩家。

\begin{itemize}
\item B->C \texttt{大麦} , \texttt{永乐}, \texttt{东方票务} 等
\item C->C \texttt{西十区}, \texttt{票牛}, \texttt{牛魔王} 等
\end{itemize}

其实,我更愿意用 \texttt{一级市场} 和 \texttt{二级市场} 这样的概念尝试去理解这两类模式。

\subsection*{各自的优势和劣势}
\label{sec:orgheadline1}

\begin{itemize}
\item B->C (一级市场)

\begin{itemize}
\item 优势

\begin{itemize}
\item 规模化(商品的差异化越小,越容易规模化), 从而成本低
\item 服务稳定 (不可控的因素很少, 直接对接主办方和用户,解决掉常规票务销售的渠道问题)
\end{itemize}

\item 劣势

\begin{itemize}
\item 作为计划经济的实现,定价会脱离实际, 也部分提供了投机的空间,促进了票务市场的紊乱
\end{itemize}
\end{itemize}

\item C->C (二级市场)

\begin{itemize}
\item 优势

\begin{itemize}
\item 提供了反应票券实际价值的可能性, 给上游更大的利益空间
\item 增加了票券流通的可能性,让有意愿的用户更有可能规范的买到
\item 相对规范的实施票务价值的更大化
\end{itemize}

\item 劣势

\begin{itemize}
\item 服务不稳定,上下游资源不可控
\item 交易成本高
\item 规范化程度不高
\end{itemize}
\end{itemize}
\end{itemize}

\section*{C->C 框架下的可能价值 (adding value)}
\label{sec:orgheadline5}

\begin{quote}
既然存在问题,那么解决问题就能带来相应的价值。
\end{quote}

\subsection*{问题分析}
\label{sec:orgheadline3}
\begin{itemize}
\item 服务不稳定
\begin{itemize}
\item 库存 (多-沉没成本, 少-跳票, 提前准备-销不出去的风险, 先销后采-成本的风险)
\item 定价 (信息不够透明,凭借猜测和感觉定价,交易双方都不满意)
\item 保障 (对商户的库存,成本的保障, 对用户的票品,价格的保障)
\end{itemize}

\item 交易成本高
\begin{itemize}
\item 信任的成本 (小商家,小平台的公信力, 无担保, 无规范)
\item 沟通的成本
\item 票务流通成本 (物流, 时间等)
\item 交易环节的成本 (定价,销售,采购等每一步的不确定性)
\end{itemize}

\item 规范化
\begin{itemize}
\item 还不清楚规范好了应该是什么样,暂时我只能类比淘宝,我觉得它已经比较靠谱了
\end{itemize}

\item 保障
\begin{itemize}
\item 对卖家的保障  (滞销,资金, 定价)
\item 对买家的保障 (真假,价格保护, 退换等)
\end{itemize}
\end{itemize}

\subsection*{针对问题,可能存在的价值}
\label{sec:orgheadline4}

\begin{quote}
每一个可能价值都会是方向,你创造了价值,才能分一杯羹,才能活下去。
\end{quote}

\begin{itemize}
\item 整体价值
\begin{itemize}
\item 交易行为规范化
\item 交易成本降低
\item 行为保障 (系统性抵御风险,类似保险思路)
\end{itemize}

\item 对上游的价值
\begin{itemize}
\item 提供公允的商品价值, 增加收益或降低沉没成本
\item 联合众多上游(平台化思路),提供资金,库存相关服务 (风险)
\item 减少中间商, 降低成本
\item 相对精准的营销
\end{itemize}

\item 对下游的价值
\begin{itemize}
\item 增加以合理价格购买到的可能性
\item 通过相对公允和流通的市场,增加票品商品化的收益或者降低沉没成本
\end{itemize}
\end{itemize}


\section*{差异化的方向}
\label{sec:orgheadline11}

\begin{quote}
每一个行为一定都有各自的代价和收益。
\end{quote}

\subsection*{用户服务}
\label{sec:orgheadline6}
\begin{itemize}
\item 售前
\begin{description}
\item[{交易决策}] 内容的好坏(评论),商品的质量(位置,距离), 信息服务(演出信息,演员信息)
\item[{商品发现}] 相关,热门,别人看过评论过,用户分类,精准推荐等
\item[{咨询,交流}] 客服咨询,用户间交流
\item[{场景化销售}] 情侣看什么,亲子看什么,悬疑看什么,打发周末看什么,刺激一下看什么
\end{description}

\item 售中
\begin{description}
\item[{商品流转信息}] 信息透明,增加信任感, 提前预知风险(很久都没备票完成...)
\item[{真假保障}] 赔付
\item[{价格保障}] 一大部分觉得被骗的原因是因为和实际价值不符,信息不对称, 可以曝光交易行为,可以在浮动过大的情况下适当赔付,同时,服务是可以收费的。
\end{description}

\item 售后
\begin{itemize}
\item 退换货
\item 转让
\end{itemize}
\end{itemize}

\subsection*{卖家服务}
\label{sec:orgheadline7}

我觉得这里主要是数据服务, 收集交易行为,提前提供参考,甚至平台提供保险保障。

\begin{description}
\item[{资金}] 集资采购, 分摊风险
\item[{库存}] 提前采购
\item[{定价}] 指导定价
\item[{精准销售}] 利用平台用户信息
\end{description}

\subsection*{保障}
\label{sec:orgheadline8}
\begin{itemize}
\item 真假
\item 滞销
\item 价格
\item 跳票
\end{itemize}

\subsection*{定价}
\label{sec:orgheadline9}
\begin{description}
\item[{收集交易行为}] 平台产生的,用户录入的,其他平台抓取的,等等
\item[{提供价格模型}] 演出,话剧,热门,常规,分门别类尽可能的给出详细的定价参考
\end{description}

\subsection*{行业标准化}
\label{sec:orgheadline10}

\begin{description}
\item[{推进电子票}] 降低转让,交易的成本
\item[{规范交易行为}] 真假票验证,票券位置信息,不能无票售卖,透明整体票盘分布信息(多少个人手上,多少主办手上之类) ...
\item 其他?
\end{description}
\end{document}
