% Created 2016-06-29 Wed 18:46
\documentclass[presentation]{beamer}
\usepackage[no-math]{fontspec}
\usepackage[BoldFont,SlantFont,AutoFakeBold=true,AutoFakeSlant=true]{xeCJK}
\setCJKmainfont[BoldFont=FandolSong-Bold.otf,ItalicFont=FandolKai-Regular.otf]{FandolSong-Regular.otf}
\setCJKsansfont[BoldFont=FandolHei-Bold.otf]{FandolHei-Regular.otf}
\setCJKmonofont{FandolFang-Regular.otf}
\usefonttheme[stillsansseriflarge,stillsansserifsmall]{serif}
\usepackage{graphicx}
\usepackage{xcolor}
\usepackage{listings}
\defaultfontfeatures{Mapping=tex-text}
\usepackage{geometry}
\usepackage{verbatim}
\usepackage{fixltx2e}
\usepackage{longtable}
\usepackage{float}
\usepackage{wrapfig}
\usepackage{rotating}
\usepackage[normalem]{ulem}
\usepackage{amsmath}
\usepackage{marvosym}
\usepackage{wasysym}
\usepackage{amssymb}
\usepackage{hyperref}
\setlength{\parindent}{0in}
\tolerance=1000
\usetheme{default}
\author{enzo liu}
\date{2016-01-11}
\title{用org-page尝试写点东西}
\hypersetup{
 pdfauthor={enzo liu},
 pdftitle={用org-page尝试写点东西},
 pdfkeywords={},
 pdfsubject={},
 pdfcreator={Emacs 25.0.94.1 (Org mode 8.3.4)},
 pdflang={English}}
\begin{document}

\maketitle

\section*{背景}
\label{sec:orgheadline1}

近段时间一直喜欢自己用 \texttt{org-mode} 写点东西。业务上的文档,个人的笔记等。

没有深入的了解过 \uline{publish} 在 \texttt{org-mode} 中怎么使用,正巧遇到了 \href{https://github.com/kelvinh/org-page}{\texttt{org-page}}, 上手很容易,就准备把自己之前的小blog给推了,用它来记录点东西。

\subsection*{test}
\label{sec:orgheadline1}

testse

\section*{安装}
\label{sec:orgheadline1}

如文档上所说,使用 \texttt{melpa} 可以很容易就安装成功。 略微配置一下文件夹地址,哪些文件夹不要发布到网站上,使用的 \texttt{branch} 的名称即可。

\section*{迁移}
\label{sec:orgheadline1}

虽然过去的 \texttt{blog} 写的很幼稚, 但是还有有一点 \sout{历史} 意义的。

使用 \texttt{haskell} 写的 \texttt{pandoc} 就是文档转换界的一把金光闪闪的瑞士军刀。

\lstset{frame=single,backgroundcolor=\color[gray]{0.95},identifierstyle=\ttfamily,keywordstyle=\color[rgb]{0,0,1},commentstyle=\color[rgb]{0.133,0.545,0.133},stringstyle=\color[rgb]{0.627,0.126,0.941},basicstyle=\scriptsize,extendedchars=true,breaklines=true,prebreak=\raisebox{0ex}[0ex][0ex]{\ensuremath{\hookleftarrow}},columns=fixed,keepspaces=true,showstringspaces=false,numbers=left,numberstyle=\tiny,language=bash,label= ,caption= ,captionpos=b}
\begin{lstlisting}
find . -name \*.md -type f -exec pandoc  -f markdown -t org -o {}.org {} \;
\end{lstlisting}

一行搞定, 然后手动加一些 \texttt{org-page} 需要的 \texttt{header} 就可以了。

\section*{github pages}
\label{sec:orgheadline3}

\subsection*{test}
\label{sec:orgheadline2}
\begin{frame}[fragile,label={sec:orgheadline1}]{test}
 原先就是使用 \texttt{jekyll} 托管在 \texttt{github page} 上, 现在只需要把 \texttt{master} 分支的文件替换成 \texttt{org-page} 生成的文件就可以了。

\lstset{frame=single,backgroundcolor=\color[gray]{0.95},identifierstyle=\ttfamily,keywordstyle=\color[rgb]{0,0,1},commentstyle=\color[rgb]{0.133,0.545,0.133},stringstyle=\color[rgb]{0.627,0.126,0.941},basicstyle=\scriptsize,extendedchars=true,breaklines=true,prebreak=\raisebox{0ex}[0ex][0ex]{\ensuremath{\hookleftarrow}},columns=fixed,keepspaces=true,showstringspaces=false,numbers=left,numberstyle=\tiny,language=bash,label= ,caption= ,captionpos=b}
\begin{lstlisting}
git checkout master
find . -name \* | grep -v ".git" | xargs rm -rf
git add . && git commit -m “clear all"
git checkout source
\end{lstlisting}

然后在 \texttt{emacs} 里执行

\lstset{frame=single,backgroundcolor=\color[gray]{0.95},identifierstyle=\ttfamily,keywordstyle=\color[rgb]{0,0,1},commentstyle=\color[rgb]{0.133,0.545,0.133},stringstyle=\color[rgb]{0.627,0.126,0.941},basicstyle=\scriptsize,extendedchars=true,breaklines=true,prebreak=\raisebox{0ex}[0ex][0ex]{\ensuremath{\hookleftarrow}},columns=fixed,keepspaces=true,showstringspaces=false,numbers=left,numberstyle=\tiny,language=Lisp,label= ,caption= ,captionpos=b}
\begin{lstlisting}
(op/do-publication t nil nil t)
\end{lstlisting}

That's done.
\end{frame}
\end{document}
