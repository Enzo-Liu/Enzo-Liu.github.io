% Created 2017-04-18 Tue 23:08
% Intended LaTeX compiler: pdflatex
\documentclass[11pt,a4paper]{article}
\usepackage{fontspec}
\usepackage{xeCJK}
\setCJKmainfont[BoldFont=FandolSong-Bold.otf,ItalicFont=FandolKai-Regular.otf]{FandolSong-Regular.otf}
\setCJKsansfont[BoldFont=FandolHei-Bold.otf]{FandolHei-Regular.otf}
\setCJKmonofont{FandolFang-Regular.otf}
\usepackage{graphicx}
\usepackage{xcolor}
\usepackage{listings}
\defaultfontfeatures{Mapping=tex-text}
\usepackage{geometry}
\usepackage{verbatim}
\usepackage{fixltx2e}
\usepackage{longtable}
\usepackage{float}
\usepackage{wrapfig}
\usepackage{rotating}
\usepackage[normalem]{ulem}
\usepackage{amsmath}
\usepackage{marvosym}
\usepackage{wasysym}
\usepackage{amssymb}
\usepackage{hyperref}
\usepackage{parskip}
\setlength{\parindent}{15pt}
\usepackage{indentfirst}
\geometry{a4paper, textwidth=6.5in, textheight=10in,
            marginparsep=7pt, marginparwidth=.6in}
\tolerance=1000
\pagestyle{empty}
\author{enzo liu}
\date{2017-04-18 Tue}
\title{读完 SICP, 我到底学会了啥}
\hypersetup{
 pdfauthor={enzo liu},
 pdftitle={读完 SICP, 我到底学会了啥},
 pdfkeywords={sicp, learning},
 pdfsubject={what do I learn from SICP},
 pdfcreator={Emacs 25.1.1 (Org mode 9.0.5)},
 pdflang={English}}
\begin{document}

\maketitle
最近遇到了好几位同学找我推荐本书学习计算机。一般情况下,我就直接报 \texttt{SICP} 。
作为一本基础入门的书,介绍了众多复杂的概念却还讲的异常简单。尤为喜爱作者公开课中最常讲的一句话 \texttt{See, it's no magic here.}

真要说好在哪里的话,其实不容易。就借机罗列一下我学到的东西,觉得有价值的,可以去看看书,听听课体会一下:)

\textbf{免责声明: 以下为一家之言,仅供参考。}

\section*{抽象}
\label{sec:org03f49bf}
怎么说呢,我觉得,程序本质就在于抽象, 而抽象, 又在于真实而复杂的世界不是一个正常人可以直接理解的。
首章直接引用了 <An Essay Concerning Human Understanding>
\begin{quote}
\begin{enumerate}
\item Combining several simple ideas into one compound one, and thus all complex ideas are made.
\item The second is bringing two ideas, whether simple or complex, together, and setting them by one another so as to take a view of them at once, without uniting them into one, by which it gets all its ideas of relations.
\item The third is separating them from all other ideas that accompany them in their real existence: this is called abstraction, and thus all its general ideas are made.
\end{enumerate}
\end{quote}
翻译不好,自行理解...下面回到正题。

\section*{求值模型}
\label{sec:org408d565}
计算机如何求解一个简单表达式的. (+ a b) \sout{(a + b)}

\section*{表达式组合}
\label{sec:org0917cb7}

\section*{函数/过程(Procedures)组合}
\label{sec:org394bfc3}
\end{document}