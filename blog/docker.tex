% Created 2016-07-15 Fri 11:39
\documentclass[11pt,a4paper]{article}
\usepackage{fontspec}
\usepackage{xeCJK}
\setCJKmainfont[BoldFont=FandolSong-Bold.otf,ItalicFont=FandolKai-Regular.otf]{FandolSong-Regular.otf}
\setCJKsansfont[BoldFont=FandolHei-Bold.otf]{FandolHei-Regular.otf}
\setCJKmonofont{FandolFang-Regular.otf}
\usepackage{graphicx}
\usepackage{xcolor}
\usepackage{listings}
\defaultfontfeatures{Mapping=tex-text}
\usepackage{geometry}
\usepackage{verbatim}
\usepackage{fixltx2e}
\usepackage{longtable}
\usepackage{float}
\usepackage{wrapfig}
\usepackage{rotating}
\usepackage[normalem]{ulem}
\usepackage{amsmath}
\usepackage{marvosym}
\usepackage{wasysym}
\usepackage{amssymb}
\usepackage{hyperref}
\usepackage{parskip}
\setlength{\parindent}{15pt}
\usepackage{indentfirst}
\geometry{a4paper, textwidth=6.5in, textheight=10in,
            marginparsep=7pt, marginparwidth=.6in}
\tolerance=1000
\pagestyle{empty}
\author{enzo liu}
\date{2016-06-30 Thu}
\title{docker的使用}
\hypersetup{
  pdfkeywords={<TODO: insert your keywords here>},
  pdfsubject={<TODO: insert your description here>},
  pdfcreator={Emacs 25.0.94.1 (Org mode 8.2.10)}}
\begin{document}

\maketitle
前段时间,公司意外断电,导致内网服务器硬盘数据损坏, 之前纯手工打造的内网环境全跪了。欲哭无泪,于是就不哭了,借这个机会体验一把 \verb~docker~, 这里记录一下使用中遇到的一些状况。

\section*{环境介绍}
\label{sec-1}

原来的内网环境中,包含了

\begin{quote}


\begin{description}
\item[{jenkins}] 用于 \verb~CI~
\item[{artifactory}] 用于内网 \verb~maven~ 仓库的托管
\item[{gitlab}] 内网 `git` 仓库
\item[{\href{https://github.com/code4craft/blackhole}{blackhole}}] \verb~亿华~ 出品,用于内网dns劫持
\item[{nginx}] 内网各种web服务的汇总代理, 偶尔用于劫持外部网页,做些自动化的小后门
\end{description}

以及内网的文件服务器,上传服务器等
\end{quote}

去年8月份,公司刚成立,一个人瞎倒腾了3-4天才大致搞定了这些。

\section*{docker搭建的过程}
\label{sec-2}

\begin{itemize}
\item 用 \verb~docker pull~ \verb~docker run~
\end{itemize}

以 \verb~jenkins~ 为例,找了一个 \verb~docker image~
\lstset{frame=single,backgroundcolor=\color[gray]{0.95},identifierstyle=\ttfamily,keywordstyle=\color[rgb]{0,0,1},commentstyle=\color[rgb]{0.133,0.545,0.133},stringstyle=\color[rgb]{0.627,0.126,0.941},basicstyle=\scriptsize,extendedchars=true,breaklines=true,prebreak=\raisebox{0ex}[0ex][0ex]{\ensuremath{\hookleftarrow}},columns=fixed,keepspaces=true,showstringspaces=false,numbers=left,xleftmargin=.25in,xrightmargin=.25in,numberstyle=\tiny,language=shell,label= ,caption= }
\begin{lstlisting}
docker pull jenkins
docker run -d -p 49001:8080 -v $PWD/jenkins:/var/jenkins_home -t jenkins
\end{lstlisting}

done!

\begin{itemize}
\item 用 \textasciitilde{}docker-compose\textasciitilde{}
\end{itemize}

我使用的是这种方式 \sout{因为搜索到的第一个介绍是这样用的} , 示例的 compose file 如下:

\lstset{frame=single,backgroundcolor=\color[gray]{0.95},identifierstyle=\ttfamily,keywordstyle=\color[rgb]{0,0,1},commentstyle=\color[rgb]{0.133,0.545,0.133},stringstyle=\color[rgb]{0.627,0.126,0.941},basicstyle=\scriptsize,extendedchars=true,breaklines=true,prebreak=\raisebox{0ex}[0ex][0ex]{\ensuremath{\hookleftarrow}},columns=fixed,keepspaces=true,showstringspaces=false,numbers=left,xleftmargin=.25in,xrightmargin=.25in,numberstyle=\tiny,language=docker-compose,label= ,caption= }
\begin{lstlisting}
version: '2'

services:
  oss:
    restart: always
    image: jfrog-docker-reg2.bintray.io/jfrog/artifactory-oss:latest
    volumes:
      - /var/opt/jfrog/artifactory/backup:/var/opt/jfrog/artifactory/backup:Z
      - /var/opt/jfrog/artifactory/logs:/var/opt/jfrog/artifactory/logs:Z
      - /var/opt/jfrog/artifactory/data:/var/opt/jfrog/artifactory/data:Z
      - /var/opt/jfrog/artifactory/etc:/var/opt/jfrog/artifactory/etc:Z
    ports:
      - "10081:8081"
\end{lstlisting}

一共花了 \textbf{半天} 时间,完成了 \verb~gitlab~, \verb~gitlab-runner~, \verb~spark~, \verb~artifactory~ 的搭建, 主要的时间还是花在了下载 \textasciitilde{}docker image\textasciitilde{} 的过程中。


\section*{遇到的问题}
\label{sec-3}

\begin{itemize}
\item 没有 \verb~image~ 和 \verb~container~ 的概念

第一次接触,对于其中的概念没有理解清楚就开始用,直接导致了后面的问题。

\begin{itemize}
\item 容器的数据没有持久化

部署的容器完全以搜索到的介绍为准,让我 \verb~mount~ 数据目录的,我就操作。没让我 \verb~mount~ 的,我就直接启动。于是,在一次重启之后, \verb~artifactory~ 内下载下来的所有 \verb~jar~ 包都丢了。
\end{itemize}

\item 权限

\verb~mount~ 进去的数据文件的权限默认是 \verb~root~ (\texttt{其实和使用的image相关,看它是怎么设置目录权限的,如果没有管你mount进去的文件,就默认为root}), 如果想手工调整,看该镜像是否提供了 \verb~entry point~ 用于修改。或者自己可以基于该 \verb~image~ 重编一个符合自己需求的镜像。

\item 环境变量

在安装完成 \verb~gitlab-runner~ (类似于 \verb~jenkins~ 的一个 \verb~ci~ 工具), 我希望以我指定的 \verb~java~ 和 \verb~maven~ 来执行打包等操作。于是将这些文件 \verb~mount~ 进去之后, 需要设定一下环境变量来使用指定路径下的软件。 \verb~compose file~ 中有一个 \verb~directive~ 叫 \verb~environments~ 可以完成。

\item 指定host的解析

由于我们的mvn, ci都是在内网,并且手动绑定host来指定到某个ip, 所以也需要手工设置容器内的host。通过 \verb~compose file~ 中的 \verb~extra_hosts~ 即可制定。
\end{itemize}

\section*{题外话}
\label{sec-4}

遇到的磁盘坏了的坑,于是搭了 \verb~raid-1~, 脚本每天备份,备份数据文件夹以及对应的 \verb~docker compose file~ 。以后 \sout{服务器跪了} 再重新搭建应该更快了.

\section*{总结}
\label{sec-5}

搭建,部署,启动 \verb~docker~ 真的都是 \verb~好快~ , \verb~好快~ , \verb~好快~  ...

\textbf{科技改变生活}
% Emacs 25.0.94.1 (Org mode 8.2.10)
\end{document}
