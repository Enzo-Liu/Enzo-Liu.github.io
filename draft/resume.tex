% Created 2018-05-11 五 15:08
% Intended LaTeX compiler: pdflatex
\documentclass[11pt,a4paper]{article}
\usepackage{fontspec}
\usepackage{xeCJK}
\setCJKmainfont[BoldFont=FandolSong-Bold.otf,ItalicFont=FandolKai-Regular.otf]{FandolSong-Regular.otf}
\setCJKsansfont[BoldFont=FandolHei-Bold.otf]{FandolHei-Regular.otf}
\setCJKmonofont{FandolFang-Regular.otf}
\usepackage{graphicx}
\usepackage{xcolor}
\usepackage{listings}
\defaultfontfeatures{Mapping=tex-text}
\usepackage{geometry}
\usepackage{verbatim}
\usepackage{fixltx2e}
\usepackage{longtable}
\usepackage{float}
\usepackage{wrapfig}
\usepackage{rotating}
\usepackage[normalem]{ulem}
\usepackage{amsmath}
\usepackage{marvosym}
\usepackage{wasysym}
\usepackage{amssymb}
\usepackage{hyperref}
\usepackage{parskip}
\setlength{\parindent}{15pt}
\usepackage{indentfirst}
\geometry{a4paper, textwidth=6.5in, textheight=10in,
            marginparsep=7pt, marginparwidth=.6in}
\tolerance=1000
\pagestyle{empty}
\author{enzo liu}
\date{2017-11-29 Wed}
\title{如何写一个好的简历}
\hypersetup{
 pdfauthor={enzo liu},
 pdftitle={如何写一个好的简历},
 pdfkeywords={resume, working},
 pdfsubject={如何写一个好的简历},
 pdfcreator={Emacs 25.3.1 (Org mode 9.1.7)},
 pdflang={English}}
\begin{document}

\maketitle
\begin{quote}
背景: 我的简历写的并不好, 最近在招人,尝试从招人的角度分析一下,怎么样算是一个好的简历,以此为契机更新一下自己的。
\end{quote}

first thing first, 招人来是干活的。干活重点是能把事情做成,也就是分析问题解决问题的能力。招人的关键,就是评估对方(当前/将来)解决问题的能力。而简历,就是用于佐证你的评估的。

从某种角度而言,简历很类似一个产品,卖的并不是功能,而是解决方案/服务。\footnote{也是 \texttt{feature} 和 \texttt{function} 的区别。我认为区别在于, \texttt{关注点在于效果, 场景化, 实用} , 更具体的解释我暂时也说不出来。} 所以简历的重点也应该是陈诉所提供的解决方案,并以背景和经验加以证明。

以下以研发线举例说明。

\section*{研发工作职责}
\label{sec:orgb0f1f92}
\texttt{专业}, \texttt{管理协调}\footnote{价值主要体现在业务价值上?}

\begin{enumerate}
\item 做为一个专业领域,有一定的门槛,能把东西做出来一个主要职责\footnote{专业岗位的核心在于技能?质量也是评价做出来了的一个标准。}
\item 涉及到做事,很重要的一点就是做正确的事(或者说是最重要的一点)
\item 一旦协作的人/角色多起来,需要有一定的协调管理能力
\item 一旦事情多起来,需要有区分重要性的能力
\item 一旦事情的复杂度上升,需要有分析拆解的能力 \footnote{拆解的核心就是定义边界和职责,只要不相互牵连太多,在一个简单领域内的问题一般都有办法解决}
\item 专业度更高的领域需要更多的特定知识和经验
\begin{itemize}
\item 比如某些特定的算法
\item 数据分析的某些应用
\item ...
\end{itemize}
\item 质量保障,系统保障等其他专业研发领域
\end{enumerate}

\subsection*{初级工程师}
\label{sec:org2ee4b2e}
\begin{enumerate}
\item 协助项目开发, 负责日常功能开发
\item 对整体业务\&项目有一定了解
\begin{itemize}
\item 对周边项目关系,各自职责分工有一定认识
\item 对用户,各类用户诉求,产品价值有一定了解
\end{itemize}
\item 在自己的领域范围内,对于正确的事情有自己的判断
\item 能够突破自己的 scope 做一定的事情
\end{enumerate}

\subsection*{中级工程师}
\label{sec:orgca34385}

\subsection*{工作难点}
\label{sec:org324539b}
\begin{itemize}
\item 功能开发

\item 业务架构
\begin{itemize}
\item 模糊的产品概念如何实施
\item 复杂的业务系统如何实现
\item 如何拆分不同的业务领域
\item 怎样给不同的业务定级
\end{itemize}

\item 系统架构
\begin{itemize}
\item 为什么要分层

\item 如何支持业务的拆分

\item 为什么要容错

\item 为什么要...
\end{itemize}

\item 产品

\item
\end{itemize}
\end{document}