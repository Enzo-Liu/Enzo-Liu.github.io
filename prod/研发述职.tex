% Created 2018-05-28 Mon 23:00
% Intended LaTeX compiler: pdflatex
\documentclass[presentation, bigger]{beamer}
\usepackage[no-math]{fontspec}
\usepackage[BoldFont,SlantFont,AutoFakeBold=true,AutoFakeSlant=true]{xeCJK}
\setCJKmainfont[BoldFont=FandolSong-Bold.otf,ItalicFont=FandolKai-Regular.otf]{FandolSong-Regular.otf}
\setCJKsansfont[BoldFont=FandolHei-Bold.otf]{FandolHei-Regular.otf}
\setCJKmonofont{FandolFang-Regular.otf}
\usefonttheme[stillsansseriflarge,stillsansserifsmall]{serif}
\usepackage{graphicx}
\usepackage{xcolor}
\usepackage{listings}
\defaultfontfeatures{Mapping=tex-text}
\usepackage{geometry}
\usepackage{verbatim}
\usepackage{fixltx2e}
\usepackage{longtable}
\usepackage{float}
\usepackage{wrapfig}
\usepackage{rotating}
\usepackage[normalem]{ulem}
\usepackage{amsmath}
\usepackage{marvosym}
\usepackage{wasysym}
\usepackage{amssymb}
\usepackage{hyperref}
\setlength{\parindent}{0in}
\tolerance=1000
\usetheme{metropolis}
\author{刘恩泽}
\date{2018-05-28}
\title{研发岗位述职}
\hypersetup{
 pdfauthor={刘恩泽},
 pdftitle={研发岗位述职},
 pdfkeywords={},
 pdfsubject={},
 pdfcreator={Emacs 26.1 (Org mode 9.1.13)},
 pdflang={English}}
\begin{document}

\maketitle
\begin{frame}{目录}
\tableofcontents
\end{frame}


\section{研发能力}
\label{sec:org56a8a25}
\begin{frame}[label={sec:org72f7fe4}]{通用能力}
\begin{description}
\item[{问题解决}] 4
\begin{itemize}
\item iptables, 搜索, hanlp, 相关推荐, spark, dubbo, docker, solr-cloud\ldots{}
\end{itemize}
\item[{学习能力}] 3
\begin{itemize}
\item 标准: 总结提炼,帮助他人
\item \url{http://git.pn.com/piaoniu/piaoniu.wiki/tree/master}
\item 真实标准: 掌握新事物/概念的能力,抽象和泛化的能力
\end{itemize}
\end{description}
\end{frame}

\begin{frame}[label={sec:orgf423d39}]{专业能力}
\begin{description}
\item[{业务开发}] 4
\begin{itemize}
\item 商家系统, 票务系统, 分销系统, 等等
\item 较高的研发效率, 以及系统不能做的事情较少
\end{itemize}
\item[{业务需求}] 3
\begin{itemize}
\item 主办产品没有很成功\ldots{}
\end{itemize}
\item[{系统架构}] 4
\begin{itemize}
\item 知晓好的架构标准/实践
\item 在当前的环境尽可能的选择合适的方式
\end{itemize}
\end{description}
\end{frame}

\begin{frame}[label={sec:org48e65b3}]{影响力}
\begin{description}
\item[{知识传播}] 3
\begin{itemize}
\item 内部总结,团队分享,公司分享
\end{itemize}
\item[{人才培养}] 3
\begin{itemize}
\item 人才梯队建设不足
\end{itemize}
\end{description}
\end{frame}

\section{岗位职责}
\label{sec:orgf8ee7e8}
\begin{frame}[label={sec:org647056e}]{岗位}
\begin{itemize}
\item 行业线研发 Leader
\end{itemize}
\end{frame}
\begin{frame}[label={sec:orgdbf93f1}]{职责}
\begin{itemize}
\item 负责带领行业线研发,配合业务方,达成相关目标
\end{itemize}
\end{frame}
\begin{frame}[label={sec:org0ebd317}]{职责 II}
\begin{itemize}
\item 根据公司发展战略, 拟定中长期研发计划, 把握研发方向
\item 指导并监督研发部门执行公司研发战略和年度研发计划
\item 控制产品开发进度, 调整计划
\item 组建优秀的产品研发团队, 审核及培训考核有关技术人员
\end{itemize}
\end{frame}

\section{个人评价}
\label{sec:orge325ff5}
\begin{frame}[label={sec:org98e09da}]{优势}
\begin{itemize}
\item 问题解决 (攻关型)
\item 分析
\item 研发落实
\end{itemize}
\end{frame}

\begin{frame}[label={sec:org4edfe47}]{劣势}
\begin{itemize}
\item 业务方向
\item 团队管理
\end{itemize}
\end{frame}
\end{document}