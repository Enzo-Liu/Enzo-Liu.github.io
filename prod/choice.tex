% Created 2018-05-06 Sun 06:44
% Intended LaTeX compiler: pdflatex
\documentclass[11pt,a4paper]{article}
\usepackage{fontspec}
\usepackage{xeCJK}
\setCJKmainfont[BoldFont=FandolSong-Bold.otf,ItalicFont=FandolKai-Regular.otf]{FandolSong-Regular.otf}
\setCJKsansfont[BoldFont=FandolHei-Bold.otf]{FandolHei-Regular.otf}
\setCJKmonofont{FandolFang-Regular.otf}
\usepackage{graphicx}
\usepackage{xcolor}
\usepackage{listings}
\defaultfontfeatures{Mapping=tex-text}
\usepackage{geometry}
\usepackage{verbatim}
\usepackage{fixltx2e}
\usepackage{longtable}
\usepackage{float}
\usepackage{wrapfig}
\usepackage{rotating}
\usepackage[normalem]{ulem}
\usepackage{amsmath}
\usepackage{marvosym}
\usepackage{wasysym}
\usepackage{amssymb}
\usepackage{hyperref}
\usepackage{parskip}
\setlength{\parindent}{15pt}
\usepackage{indentfirst}
\geometry{a4paper, textwidth=6.5in, textheight=10in,
            marginparsep=7pt, marginparwidth=.6in}
\tolerance=1000
\pagestyle{empty}
\author{enzo liu}
\date{\today}
\title{}
\hypersetup{
 pdfauthor={enzo liu},
 pdftitle={},
 pdfkeywords={},
 pdfsubject={},
 pdfcreator={Emacs 26.1 (Org mode 9.1.12)},
 pdflang={English}}
\begin{document}

\tableofcontents

\section{我会考虑几个点}
\label{sec:orgd8296c0}
\begin{enumerate}
\item 在花旗的什么部门
\begin{itemize}
\item (加分) 交易系统
\item (加分) 风控系统
\item 如果内部系统我就基本不考虑了
\end{itemize}
\item 考拉的哪个部门
\begin{itemize}
\item 如果内部系统也建议不要考虑
\item 电商部门的话,用户侧偏订单交易,运营活动类有一定复杂度。后台仓配逻辑如果要做提高效率可能会有一定的复杂度。其他的业务部门就需要再看了.
\end{itemize}
\item 从公司的发展的可能性(公司成长性)来说, 我认为考拉 > 花旗
\begin{itemize}
\item 个人的发展大概率会随着公司成长
\end{itemize}
\item 如果从技术能力来说
\begin{itemize}
\item 整体来说:
\begin{itemize}
\item 业务分析
\item (复杂业务系统的) 业务架构
\begin{itemize}
\item 电商的业务系统还算是相对复杂的系统
\end{itemize}
\item (高并发,高可用) 系统架构
\item 分布式系统
\end{itemize}
\item 从这些角度来评估的话
\begin{itemize}
\item 金融的核心系统门槛肯定会更高(对成长更有利),如果不是,(感觉)考拉的整体技术氛围应该优于花旗
\item 除了业务分析能力能日常练习,其他如果自己天赋不是特别好,没有相关系统练手,不太可能
\end{itemize}
\item 技术的发展上
\begin{itemize}
\item 任何一个相对复杂的系统能吃透我觉得都 ok.
\item 想学习金融知识主要靠自己,靠在花旗我觉得没啥用\ldots{}
\begin{itemize}
\item 可以考一个 cfa,比靠花旗的研发背景靠谱
\end{itemize}
\item 英语不管在哪里都很重要
\item 其他专项的能力,比如大数据,比如算法,搜索,推荐,分词,比如分布式系统,比如存储,比如 AI, Machine Learning 之类,都有一部分门槛,如果跨不过去,入门都很难. (自己得主动去学,靠机会非常难)
\end{itemize}
\end{itemize}
\item 待遇上, (我认为) 几 k 的差异可以忽略\ldots{}
\end{enumerate}

\section{整体来说}
\label{sec:orged68684}
如果考虑个人发展,我更建议考拉。
\begin{itemize}
\item 能力的发展我觉得更多的靠个人。环境有一部分作用,师傅有更大一部分的作用。但是主要还是个人
\item 事业的发展出了能力以外,还有很大一部分机遇的部分,我觉得考拉的可能性更大
\item 现阶段的级别,我觉得公司的发展能给你带来的分润都不会很多\ldots{}这个阶段,还是能力的提升给自己待遇提升的帮助更大. 如果花旗是交易系统等核心系统研发,我应该会选花旗
\end{itemize}


\section{关于选择}
\label{sec:orgf58528e}
\begin{enumerate}
\item 投行部算是比较核心的部门。主要业务应该是:
\begin{itemize}
\item 承揽: 对公就是承接公司的融资需求
\item 承做: 金融产品开发(对公一般就是帮企业发债,股权融资, 做 IPO 等)
\item 承销: 卖这些金融产品
\item 研发部门应该不太核心,但是能接触到金融相关概念肯定不少。
\end{itemize}
\item 考拉主站研发: 应该算是考拉较核心的部门
\end{enumerate}

\section{想出国}
\label{sec:org14771db}
\begin{itemize}
\item 公司
\begin{itemize}
\item 跨国公司国内转是一个途径
\item 直接去面试一些海外公司
\begin{itemize}
\item (基础)算法要求一般不低
\item 英语水平
\end{itemize}
\end{itemize}
\item 学校
\begin{itemize}
\item 考个 GRE
\item 找人写推荐信
\end{itemize}
\end{itemize}

\section{技术到底应该怎么学习}
\label{sec:orgf15f074}
\subsection{技术是啥}
\label{sec:org97d26c8}
解决实际问题的能力

\subsection{常见问题}
\label{sec:org972ca76}
\begin{itemize}
\item 业务问题
\begin{itemize}
\item 了解业务背景,分析问题,给出方案
\end{itemize}
\item 复杂的业务问题
\begin{itemize}
\item 和其他业务的冲突
\item 和其他问题的共性
\item 如何拆解成子问题
\end{itemize}
\item 系统问题
\begin{itemize}
\item 慢
\item 稳定(比如前几天宕机)
\item 安全
\item 逻辑出错(如何排查)
\item 研发效率(人多之后怎么办)
\item 业务架构(业务功能太多之后怎么办)
\item 系统架构(和以上都有关系)
\item 其他专项技能(比如搜索怎么做,这些问题一个门槛在能和不能,一个门槛在做好)
\end{itemize}
\item 其他难题
\begin{itemize}
\item 其实难就体现在门槛上而已,会的人少,就算是难题。
\end{itemize}
\end{itemize}

\subsection{公司对于发展的差异}
\label{sec:org73a642c}
我觉得大部分公司应该都差不多。主要有几点差异
\begin{itemize}
\item 日常培训机制(培训里,最重要的还是自己,投入度显著影响效果)
\item 发展太快, 没人处理难题, 必须顶上(公司发展超过个人发展,会强迫拔高,如果跟上,很好,跟不上,很难发展\ldots{})
\item 有厉害的团队 (可以看到别人如何做事,和别人多交流学习)
\item 一些专业的部门(比如架构组和普通业务研发,解决的问题的门槛就不一样。)
\end{itemize}

\subsection{怎么学习/提升}
\label{sec:org8d64f3f}
我的个人经历/经验:
\begin{itemize}
\item 多参与培训(会影响人的视野)
\begin{itemize}
\item 曾经在华为,听到了很多没接触过的东西.
\item 到点评去 qcon 收获也不错.
\item 我不太喜欢太具体(针对具体事务)的培训,对我的帮助不大
\end{itemize}
\item 前 5 年工作中学习很多
\begin{itemize}
\item 问题有时需要自己找,怎么提高开发效率,怎么保证上线后容易发现问题,怎么优化线上稳定性和性能,怎么找到业务的模式做的通用化
\end{itemize}
\item 这两年看书学习很多
\begin{itemize}
\item 首先要有个全盘一点的概念, 有哪些东西可以学,想学什么
\item 最前沿的读论文,其他看一些经典的书籍
\end{itemize}
\item 交流
\begin{itemize}
\item 看别人的经历,看待问题的方式方法, 解决问题的方式方法
\end{itemize}
\item 总结和思考
\begin{itemize}
\item 如果解决的问题多了,能沉淀一些方法,能找到不同模式间更通用的东西
\item 就像说日常用的框架,外貌一直在变,底子就那几种。能整理出脉络,遇到其他问题基本也都能解决。(这个我认为就是技术能力了)
\end{itemize}
\end{itemize}
\end{document}