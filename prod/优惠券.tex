% Created 2018-01-30 二 14:58
% Intended LaTeX compiler: pdflatex
\documentclass[11pt,a4paper]{article}
\usepackage{fontspec}
\usepackage{xeCJK}
\setCJKmainfont[BoldFont=FandolSong-Bold.otf,ItalicFont=FandolKai-Regular.otf]{FandolSong-Regular.otf}
\setCJKsansfont[BoldFont=FandolHei-Bold.otf]{FandolHei-Regular.otf}
\setCJKmonofont{FandolFang-Regular.otf}
\usepackage{graphicx}
\usepackage{xcolor}
\usepackage{listings}
\defaultfontfeatures{Mapping=tex-text}
\usepackage{geometry}
\usepackage{verbatim}
\usepackage{fixltx2e}
\usepackage{longtable}
\usepackage{float}
\usepackage{wrapfig}
\usepackage{rotating}
\usepackage[normalem]{ulem}
\usepackage{amsmath}
\usepackage{marvosym}
\usepackage{wasysym}
\usepackage{amssymb}
\usepackage{hyperref}
\usepackage{parskip}
\setlength{\parindent}{15pt}
\usepackage{indentfirst}
\geometry{a4paper, textwidth=6.5in, textheight=10in,
            marginparsep=7pt, marginparwidth=.6in}
\tolerance=1000
\pagestyle{empty}
\author{行业研发}
\date{2018-01-25}
\title{主办优惠券}
\hypersetup{
 pdfauthor={行业研发},
 pdftitle={主办优惠券},
 pdfkeywords={},
 pdfsubject={},
 pdfcreator={Emacs 25.3.1 (Org mode 9.1.4)},
 pdflang={English}}
\begin{document}

\maketitle
\tableofcontents


\section{优惠券说明}
\label{sec:org324e77d}

当前系统支持平台优惠券,商家优惠券。主要差异在于补贴成本由谁来承担。

\subsection{优惠形式}
\label{sec:org8c1ad2c}
系统支持立减和折扣
\begin{itemize}
\item 立减n元表示在符合优惠券使用情况下,总支付金额可立减n元
\item n折表示在符合优惠券使用条件下,总商品金额打 n 折 (当前折扣金额最多 200 元)
\end{itemize}

\subsection{优惠券使用条件}
\label{sec:org6bd610c}
\begin{enumerate}
\item 满 n 元可用
\item 在指定有效期内可用
\item 在指定客户端\footnote{ios 客户端,android, m 站, pc 等} 可用
\item 指定演出/场次可用
\item 指定商家可用
\end{enumerate}

\subsection{常见发放形式}
\label{sec:org6280de3}
\begin{enumerate}
\item 平台补贴的券
\label{sec:orga684553}
\begin{enumerate}
\item 运营活动,抽奖, 运营批量发放等
\item 会员活动,每月指定等级的会员自动发放
\item 客服补贴
\item 下单红包
\end{enumerate}

\item 商家补贴的券
\label{sec:org792f119}
\begin{enumerate}
\item 在旗舰店中领用
\end{enumerate}
\end{enumerate}

\section{旗舰店中优惠券如何配置}
\label{sec:org406bca7}
\begin{enumerate}
\item 需要主办身份
\item 主办有自己的演出
\item 在 e.piaoniu.com 中的营销体系中
\begin{enumerate}
\item 创建优惠券
\begin{itemize}
\item 店铺可用表示主办的所有演出可用
\item 商品优惠券需要指定适用的商品
\end{itemize}
\item 创建营销计划
\begin{itemize}
\item 指定营销计划的有效期
\item 选择该营销计划中可用的优惠券
\end{itemize}
\end{enumerate}
\end{enumerate}

以上操作结束后,即可在该主办的旗舰店中看到领取的模块。用户领取使用后,该部分补贴成本由主办自行承担。
\end{document}